\chapter{Introduction}\label{Chap:Intro}
Magnetohydrodynamic turbulence offers a rich variety of physical
phenomena and hence is still a field of intense research.  In
comparison to pure (incompressible) hydrodynamical turbulence, the
presence of magnetic fields introduces additional complexity to the
problem, changing the picture from the classical turbulence theory
introduced by Kolmogorov \citep{Kolmogorov41}.

Hence, MHD turbulence has long been an area of interest.
For example, a large scale background field or the field on the
largest eddy-containing scales could give rise to modifications of the
small-scale fluctuations compared to pure hydrodynamic turbulence.
Iroshnikov \cite{Iroshnikov1964} \& Kraichnan
\cite{Kraichnan1965b} firstly introduced a modified theoretical
description where only waves of opposite directions interact.
This interaction is then governed by the \alfven timescale
$\tau_A\sim l/v_A$ which is shorter than the eddy-distortion time
$\tau_l$ considered otherwise.  This introduces an additional factor
of $\tau_l/\tau_A$ in the energy-transfer time which is used in the
derivation of the hydrodynamic turbulence theory.  The \alfven effect
\citep{Iroshnikov1964,Kraichnan1965b} causes the inertial-range
scaling to differ from classical HD turbulence, effectively leading to
a more shallow spectrum of $E\sim k^{-3/2}$, rather than $E\sim k^{-5/3}$.

The theory of \alfvenic wave interaction was extended by Sridhar \&
Goldreich and Goldreich \& Sridhar 
\citep{SridharGoldreich94, SridharGoldreich95} to include interactions
of multiple \alfvenic wave modes.  It then follows that 3-mode wave
interactions do not give rise to resonances which then leads to the
conclusion of a failure of the IK theory. \hl{A first complete discussion of MHD
turbulence with resonant interactions was discussed by Galtier et. al (2000)} \cite{Galtier2000}.


The detailed analysis of MHD turbulence including resonant 4-wave
interactions gives rise to a steeper spectrum of
$E_k\propto k_{\|}^{-2}$, but also a highly anisotropic spectrum where
one has to differentiate between the perpendicular and the parallel
parts of the energy spectrum.  Cho \& Vishniac
\cite{ChoVishniac2000} and more recently also Beresnyak
\cite{Beresnyak15} conclude in their analysis that a theoretically
derived Sridhar \& Goldreich spectrum agrees with their numerical
simulations.  See the book by Biskamp \cite{BiskampBook} for a
a more detailed discussion about the effects of \alfvenic waves in
turbulent fields.

Another important phenomenon is observed when the field exhibits
\textit{magnetic helicity}. In this case it is well known that the
decay is drastically different where one observes an increase of
magnetic energy on large scales and hence, a dynamical growth of the
correlation length (see e.g. Pouquet et al., Christensson et
al., \cite{Pouquet1976,Christensson01}). This effect of an
\textit{inverse cascade} is due to the well conserved helicity during
the turbulent decay (see also \cref{Sec:hel_decay} in this work).
  
Without the presence of helicity, earlier studies by Batchelor \citep{Batchelor50},
Saffman \citep{Saffman67},  Banerjee \& Jedamzik \citep{Banerjee04} and Sethi et al. \citep{Sethi05} showed that for blue magnetic field spectra, \hl{i.e. a spectrum which rises for large Fourier modes}, the coherence length also increases but this happens 
by the damping of small-scale fluctuations, leaving only large-scale fluctuations.
In this case, the decay law for the magnetic energy and the growth rate of the coherence length
depends on the large scale spectral index of the magnetic field
fluctuations. 
Also, other numerical studies confirmed that
the peak of the magnetic spectrum moves along the large scale spectrum
while the small-scale fluctuations decay
(e.g. \cite{Campanelli07,Saveliev12,Saveliev13}). 

Recently, Brandenburg et al. \cite{Brandenburg15} suggested
that, even without helicity, the magnetic energy can increase on scales
larger than the initial integral scale, and the coherence length can
moderately grow through an effect similar to the helical case. 
But this nonhelical \textit{inverse transfer} requires high Reynolds
numbers. Hence, previous studies of nonhelical MHD turbulence decay
have not seen this effect clearly \citep{MuellerBiskamp99,
  Christensson01, Banerjee04b, Kalelkar04} whereas latest numerical studies seem
to confirm the result by Brandenburg et al.
\cite{Zrake2014,Linkmann16}. 

If the effect of the inverse transfer is proven to be universal for
large Reynolds number regimes, it would have a large impact for
magnetic field evolution, in particular, during the Early
Universe. Here, slight changes of the turbulent decay law (which is
typically a power law of the time $t$) will result in different field
strengths and coherence lengths at later epochs. For instance,
assuming a magneto-genesis scenario during the electroweak (EW) phase
transition, Wagstaff et al. \cite{Wagstaff16} \hl{showed that the
decay of the magnetic field will be too fast to explain the weak lower bounds
of the magnetic fields in the voids of galaxies as inferred from Fermi
observations of TeV Blazars} \cite{Neronov10, Taylor11}. Without the
effect of the inverse transfer the magnetic energy will decay as
$E_B \propto t^{-10/7}$ (see e.g. \cite{Banerjee04b, Sethi05}) if a
causal magnetic field spectrum with no helicity is assumed
\cite{Durrer03}. Otherwise, according to the decay law by the effect
of the inverse transfer, the magnetic energy decays as $E_B \propto
t^{-1}$, leaving strong enough present-day magnetic fields to explain
the fields in voids of galaxies \cite{Kahniashvili13}. 

Motivated by the work by Brandenburg et. al (2015) \cite{Brandenburg15},
we performed a detailed numerical investigation to test the regimes where one can
expect an efficient inverse transfer of energy to larger scales during
the decay of magnetic fields.
Unlike in the case of helical magnetic fields, where helicity is a
conserved quantity and energy is transferred to larger scales by an
inverse cascade, the physical reasoning of the non-helical
\textit{inverse transfer} is not understood (see appendix of this work
and supplement of \cite{Brandenburg15}). Our goal is to shed some light on
this phenomenon. For this reason, we mainly use the well established
Pencil Code \cite{PENCIL10}, which was also used in the original study
by Brandenburg et al. \cite{Brandenburg15}, where we vary the
Reynolds number, the Prandtl number as well as the initial spectra of
the stochastic magnetic field. 


This work starts off with a discussion of helical and nonhelical
turbulence in \cref{Chap:Turbulence}.  In \cref{Chap:Methods} we
describe the details of our numerical setup, the analysis methods and
the run-time parameters. We then present the simulation results in
\cref{Chap:Results} where we discuss the impact of the variation of
the viscosity parameter and the Prandtl number as well as the
influence of the initial spectrum on the \textit{inverse transfer}.
Furthermore, we present a run with the \textit{Zeus-MP2} Code, where
we don't see enough evidence for the \textit{inverse transfer} and
discuss how that could be linked to the numerical integration
scheme. We conclude in \cref{Chap:Conc} with the discussion
of our main findings and their impact on causally generated fields in
the early Universe questioning the effect of the inverse transfer due
to large expected Prandtl numbers. 
% \begin{itemize}
%     \item What is the subject of the study? Describe the
%         economic/econometric problem.
% 
%     \item What is the purpose of the study (working hypothesis)?
% 
%     \item What do we already know about the subject (literature
%         review)? Use citations: \textit {\citet{Gallant:87} shows that...
%         Alternative Forms of the Wald test are considered
%         \citep{Breusch&Schmidt:88}.}
% 
%     \item What is the innovation of the study?
% 
%     \item Provide an overview of your results.
% 
% 
%     \item Outline of the paper:\\
%         \textit{ The paper is organized as follows. The next section describes the
%         model under investigation. Section \ref{Sec:Data} describes the data set
%         and Section \ref{Sec:Results} presents the results. Finally, Section
%         \ref{Sec:Conc} concludes.}
% 
%     \item The introduction should not be longer than 4 pages.
%
%\end{itemize}
